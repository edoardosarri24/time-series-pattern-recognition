\section{Introduction}
\subsection{Time Series Pattern Recognition}

\begin{frame}{Introduction}
    Algoritmo per il riconoscimento di pattern in serie temporali:
    \begin{itemize}
        \item Input: Serie temporale $T$ con $N$ timestamps.
        \item Query: Pattern $Q$ di lunghezza $M$ ($M \ll N$).
        \item Output: Indice $i$ che minimizza la distanza tra $Q$ e $T$.
    \end{itemize}
    \begin{block}{Versioni}
        Sono state realizzate due implementazioni:
        \begin{itemize}
            \item Sequenziale e CPU-based.
            \item Parallela e GPU-based (CUDA).
        \end{itemize}
    \end{block}
\end{frame}

\begin{frame}{Introduction}
    \begin{block}{Dataset}
        \href{https://archive.ics.uci.edu/dataset/240/human+activity+recognition+using+smartphones}{Human Activity Recognition (HAR)} dataset: cattura il movimento di umani in gionate tipo con i sensori dello smartphone.
        \begin{itemize}
            \item Dimensioni: 6 (3 acc + 3 gyro).
            \item Samples: 2.56s a 50Hz; 10299 istanze da 128 timestamps.
        \end{itemize}
    \end{block}
    \begin{block}{Approccio}
        \textbf{Sliding Window}: per ogni istante temporale $i$ si calcola la distanza tra la query e la porzione di input della stessa lunghezza.
    \end{block}
\end{frame}

\begin{frame}{Introduction}
    \begin{block}{Input e Query}
        \begin{itemize}
            \item Input:
            \begin{itemize}
                \item Concatenazione di training e test set.
                \item Data augmentation: Più istanze aggiungendo un rumore uniforme $\epsilon=\pm0.01$.
            \end{itemize}
            \item Query:
            \begin{itemize}
                \item Sottosequenza estratta casualmente dall'input.
                \item Aggiunto rumore Gaussiano con $\mu=0, \sigma=0.01$ per avere $SAD>0$.
            \end{itemize}
        \end{itemize}
    \end{block}
\end{frame}

\begin{frame}{Introduction}
    \begin{block}{Distanza}
        \textbf{Sum of Absolute Differences} (SAD). Date $D$ dimensioni:
        \begin{equation*}
            D(i) = \sum_{m=0}^{M-1} \sum_{d=0}^{D-1} |Q[m, d] - T[i+m, d]|
        \end{equation*}
        L'obiettivo è trovare $i^* = \displaystyle\arg\min_i D(i)$.
    \end{block}
    \begin{block}{Sanitizer}
        Per trovare erorri a run-time è stato usato \href{https://github.com/edoardosarri24/time-series-pattern-recognition/blob/3544216290ac2ef624417e489b20ebed9bae3b58/exec/AUBsanitizer.sh}{AUBSan}
    \end{block}
\end{frame}