\section{Introduction}

\begin{frame}{Introduction}
    \begin{block}{Obiettivo}
        Implementazione in C++ di un algoritmo per il riconoscimento di pattern in serie temporali multivariate.
        \begin{itemize}
            \item \textbf{Input:} Serie temporale $T$ di lunghezza $N$.
            \item \textbf{Query:} Pattern $Q$ di lunghezza $M$ ($M \ll N$).
            \item \textbf{Output:} Indice $i$ che minimizza la distanza tra $Q$ e la finestra temporale in $T$.
        \end{itemize}
    \end{block}
    \begin{block}{Versioni}
        Sono state realizzate due implementazioni:
        \begin{enumerate}
            \item \textbf{Sequenziale:} CPU based.
            \item \textbf{Parallela:} GPU based (CUDA).
        \end{enumerate}
    \end{block}
\end{frame}

\begin{frame}{Introduction}
    \begin{block}{Approccio}
        Utilizzo di una \textbf{Sliding Window}: per ogni istante temporale $i$ si calcola la distanza tra la query e la porzione di input corrispondente.
    \end{block}
    \begin{block}{Metrica: SAD}
        Sum of Absolute Differences (SAD). Per dati multivariati con $D$ dimensioni:
        \[
        D(i) = \sum_{m=0}^{M-1} \sum_{k=0}^{D-1} |Q[m, k] - T[i+m, k]|
        \]
        L'obiettivo è trovare $i_{min} = \arg\min_i D(i)$.
    \end{block}
\end{frame}

\begin{frame}{Introduction}
    \begin{block}{Dataset}
        \href{https://archive.ics.uci.edu/dataset/240/human+activity+recognition+using+smartphones}{Human Activity Recognition (HAR)} dataset.
        \begin{itemize}
            \item \textbf{Dimensioni ($D$):} 6 (3 acc + 3 gyro).
            \item \textbf{Frequenza:} 50Hz.
            \item \textbf{Preprocessing:} Concatenazione di training e test set + Data Augmentation.
        \end{itemize}
    \end{block}
    \begin{block}{Input e Query}
        \begin{itemize}
            \item \textbf{Input:} Stream continuo generato concatenando i campioni e applicando un moltiplicatore con rumore uniforme.
            \item \textbf{Query:} Sottosequenza estratta casualmente dall'input con aggiunta di rumore Gaussiano ($\mu=0, \sigma=0.01$).
        \end{itemize}
    \end{block}
\end{frame}